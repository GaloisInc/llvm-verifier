\documentclass[11pt]{article}
\usepackage{amsfonts}
\usepackage{xspace}
\usepackage{url}
\usepackage{graphicx}
\usepackage[headheight=90pt,headsep=12pt,includeheadfoot,top=24pt,left=1in,footskip=24pt]{geometry}
\usepackage{verbatim}
\usepackage{fancyvrb}
\usepackage{amsmath, amsthm, amssymb}
\usepackage{fancyhdr}
\usepackage{xcolor}
\usepackage[xetex,bookmarks=true,pagebackref=true,linktocpage=true]{hyperref}
%\usepackage[style=list]{glossary}

\newcommand{\titleline}{Generating Formal Models with the LLVM Symbolic Simulator}

\hypersetup{%
   pdftitle     = \titleline,
   pdfkeywords  = {LLVM, Symbolic Simulation, Cryptography, Programming, Formal Methods, Formal Models},
   pdfauthor    = {Galois, Inc.},
   pdfpagemode  = UseOutlines
}

%\input{utils/Indexes.tex}
%\input{utils/GlossaryItems.tex}
%\input{utils/trickery.tex}

% fonts
\usepackage{fontspec}
\usepackage{xunicode}
\usepackage{xltxtra}
%\setmainfont[Mapping=tex-text]{Garamond}
\setsansfont[Mapping=tex-text]{Optima}
\usepackage{sectsty}
\allsectionsfont{\sffamily}

\newcommand{\todo}[1]{\begin{center}\framebox{\begin{minipage}{0.8\textwidth}{{\bf TODO:} #1}\end{minipage}}\end{center}}

\DefineVerbatimEnvironment{code}{Verbatim}{}

% various little text sections:
\newcommand{\draftdate}{DRAFT of \today}
% \setlength{\voffset}{-1cm}
% \setlength{\headsep}{2cm}
% \setlength{\headheight}{15.2pt}
% \renewcommand{\headrulewidth}{0pt} % no line on top
% \renewcommand{\footrulewidth}{.5pt} % line on bottom
% %\renewcommand{\chaptermark}[1]{\markboth{#1}{}}
% %\renewcommand{\sectionmark}[1]{\markright{#1}{}}
% \cfoot{}
% \fancyfoot[LE,RO]{\fancyplain{}{\textsf{\thepage}}}
% \fancyfoot[LO,RE]{\fancyplain{}{\textsf{\copyright\ 2010, Galois, Inc.}}}
% %\fancyhead[LE]{\fancyplain{}{\textsf{\draftdate}}}
% %\fancyhead[RO]{\fancyplain{}{\textsf{DO NOT DISTRIBUTE!}}}
% \fancyhead[LE]{\fancyplain{}{\textsf{\titleline}}}
% \fancyhead[RO]{\fancyplain{}{\textsf{\titleline}}}
% \fancyhead[LO,RE]{}
% \pagestyle{fancyplain}

%\makeglossary
%\makeindex

\title{\LARGE{\bf \titleline}}
\author{\\$ $\\$ $\\
        Galois, Inc.\\
        421 SW 6th Ave., Suite 300\\Portland, OR 97204}
\date{
\vspace*{2cm}$ $\\
\includegraphics[height=1in]{images/Galois_logo_blue_box.pdf}
\vspace*{2cm}$ $\\
%\draftdate
%\\
%PLEASE DO NOT DISTRIBUTE!
}

\pagestyle{fancy}
\fancyheadoffset[l]{1in}
\fancyheadoffset[r]{0.2in}
\fancyhead{}%clear
\fancyfoot{}%clear
\lhead{\sffamily\hspace{24pt}\includegraphics[height=1in]{images/Galois_logo_blue_box.pdf}}
\rhead{\sffamily\titleline\\\vspace{12pt}}
\rfoot{\sffamily\thepage}
\lfoot{\sffamily\copyright 2011 Galois, Inc.}
\renewcommand{\headrulewidth}{0pt}
\renewcommand{\footrulewidth}{1pt}

\begin{document}

\maketitle
\thispagestyle{empty}
\newpage

\section{Introduction}

This document describes how to use the Galois LLVM Symbolic Simulator's
command-line interface, \texttt{lss}, to generate a formal model of a
cryptographic algorithm compiled to LLVM from C, and to compare that
model against a reference specification to verify that the LLVM version
is correct. This brings benefits such as allowing programmers to
experiment with efficient, customized implementations of an algorithm
while retaining confidence that the changes do not affect the overall
functionality.

We assume some knowledge of C, a basic understanding of LLVM, and a
passing familiarity with cryptography. However, we do not assume
familiarity with symbolic simulation, formal modeling, or theorem
proving.  We assume that the user has installed the LLVM 2.9 toolchain.
Additionally, we assume that the user has installed the Cryptol tool set
and the ABC logic synthesis system from UC Berkeley if she wishes to
complete the equivalence checking.  Installation and configuration of
those tools is outside the scope of this tutorial, although we will be
discussing some of their use.

In the examples of interaction with the simulator and other tools, lines
beginning with a hash mark (\texttt{\#}) or short text followed by an
angle bracket (such as \texttt{abc 01>}) indicate command-line prompts,
and the following text is input provided by the user. All other
monospaced text is the output of the associated tool.

\section{Setting up the Environment}
\label{sec:setup}
 
Ensure that \texttt{clang}, \texttt{llvm-as}, and \texttt{llvm-dis} are
in your path; these tools are part of the standard LLVM distribution.

Similarly, make sure that the \texttt{lss} executable bundled with this
document is in your path.  If you intend to run the equivalence checking
part of the tutorial, \texttt{cryptol} will also need to be installed.

All source code used by this tutorial can be found within the
\texttt{lss-tutorial/code} subdirectory of the release.  There is a
Makefile in that directory that can be used to compile everything with
clang, link things together, and the \texttt{check} target can be used
to start the equivalence checking process.

\section{Symbolic Simulation}

The LLVM Symbolic Simulator takes the place of a typical
post-compilation execution environment, but makes it possible to reason
about the behavior of programs on a wide range of potential inputs,
rather than a fixed set of test vectors. A standard post-compilation
execution environment for, e.g., clang-compiled programs, runs programs
on concrete input values, producing concrete results. Symbolic
simulation works similarly, but allows inputs to take the form of
symbolic variables that represent arbitrary, unknown values. The result
is then a mathematical formula that describes the output of the program
in terms of the symbolic input variables.

Given a formula representing a program's output, we can then either
evaluate that formula with specific values in place of the symbolic
input variables to get concrete output values, or compare the formula to
another, using known mathematical transformations to prove that the two
are equivalent.

One downside of symbolic simulation, however, is that it cannot easily
handle interaction with the outside world. If a program simply takes an
input, performs some computation, and produces some output, symbolic
simulation can reliably construct a model of that computation.
Cryptographic algorithms typically fall into this category. In cases
where a program does some computation, produces some output, reads some
more input, and continues the computation based on the new information,
we either need a model of the outside world, or to assume that the input
could be completely arbitrary, and reason about what the program would
do for any possible input.

The LLVM Symbolic Simulator can evaluate simple output methods, such as
the ubiquitous \texttt{printf()}.  If the value printed depends on the
value of a symbolic input variable, a bit-level textual representation
is shown with '?' marks to denote symbolic bits.

The LLVM Symbolic Simulator provides a set of special functions for
performing operations on symbolic values, and for emitting the formal
model representing symbolic values of interest. The rest of this
tutorial will demonstrate how to use these methods to generate a formal
model of a simple AES128 implementation, and then compare this model to
a reference specification.

\section{Supplying Symbolic Input to AES128}

In the code subdirectory of the directory containing this tutorial,
there are some files and directories of note.  The file
\texttt{aes128BlockEncrypt\_driver.c} contains the driver code that sets
up the symbolic inputs to the AES128 block encrypt function;
\texttt{aes128BlockEncrypt.[ch]} contains the block encrypt function
implementation; the \texttt{sym-api} subdirectory contains copies of the
``sigfuns API'' header and implementation files; finally, the
\texttt{ref} subdirectory includes the cryptol reference specification
and equivalence checking script.

Let's start with \texttt{aes128BlockEncrypt\_driver.c}.  This code
creates a simple wrapper around the \texttt{aes128BlockEncrypt}
function.  We will use this source file as a running example, and step
through what each line means in the context of symbolic simulation.

Note that the driver includes \texttt{sym-api.h} to get access to the
special functions used to interact with the symbolic simulator.

The first two variable declarations in the \texttt{main} function are
those with the most relevance to symbolic simulation.

\begin{code}
SWord32 *pt  = lss_fresh_array_uint32(4, 0x8899aabbUL);
SWord32 *key = lss_fresh_array_uint32(4, 0x08090a0bUL);
\end{code}

These declarations each create a new array with entirely symbolic
contents, intended to be used as the plaintext and key inputs to the
block encrypt function.  Their size is fixed (4 elements of type
uint32\_t, or 128 bits), but each element is a symbolic term
representing an arbitrary 32-bit unsigned value.  The second parameter
to the \texttt{lss\_fresh\_array\_uint32} function is the initial value
for each element if this code is executed in a \emph{concrete} context
(i.e., not via \texttt{lss}).  We can ignore it for our purposes here.

The next declaration is standard C, and creates an uninitialized 128-bit
array for holding the ciphertext result.  This does not hold symbolic
values at the time of declaration, but the values stored inside it will
be symbolic if they depend on the values in \texttt{pt} or \texttt{key},
as we will expect them to.

Next, calculation of the AES128 ciphertext occurs in a typical fashion,
by calling the block encrypt function and passing both in parameters and
out parameters by pointer.

\begin{code}
aes128BlockEncrypt(pt, key, ct);
\end{code}

The next and final statement does the work of creating a formal model
from the AES128 block encrypt function.  This function instructs the
symbolic simulator to generate a formula that describes how the elements
of \texttt{ct} depend on the elements of \texttt{pt} and \texttt{key},
and then writes that formula to a file called \texttt{aes.aig}:

\begin{code}
lss_write_aiger_array_uint32(ct, 4, "aes.aig");
\end{code}

Ultimately, we want to find a formal model that describes the output of
the AES128 block encrypt function, in terms of whatever symbolic inputs
it happens to depend on. In this case, this includes every byte of the
input key and plaintext. However, for some algorithms, it could include
only a subset of the symbolic variables in the program.

The formal model that the simulator generates takes the form of an
And-Inverter Graph (AIG), which is a way of representing a boolean
function purely in terms of the logical operations ``and'' and
``not''. The simplicity of this representation makes the models easy to
reason about, and to compare to models from other sources.  However, the
same simplicity means that the model files can be very large in
comparison to the input source code.

\section{Running the Simulator}

To generate a formal model from the example described in the previous
section, we can use the \texttt{lss} command, which forms the
command-line front end of the LLVM Symbolic Simulator. At minimum, it
needs to know where to find a fully-linked LLVM bitcode containing a
\texttt{main} function.  Typing \texttt{make} inside the code
subdirectory will produce a file called \texttt{aes.bc} that meets these
criteria.

The following command will then run \texttt{lss} to create a formal
model:

\begin{code}
# lss aes.bc
\end{code}

This will result in a file called \texttt{aes.aig} that can be further
analyzed using a variety of tools, including the Galois Cryptol tool set
and the ABC logic synthesis system from UC Berkeley.

\section{Viewing the Intermediate Representations}

When working with input C source and linked LLVM bitcodes, it can be
usefult to inspect two underlying intermediate representations: LLVM
itself, and LLVM-Sym, the language to which LLVM programs are translated
by \texttt{lss}.  For example, let's say we've compiled the AES128
driver code by hand:

\begin{code}
clang -emit-llvm -I./sym-api -c aes128BlockEncrypt_driver.c \
  -o aes128BlockEncrypt_driver.bc
\end{code}

To see the LLVM assembly langugage representation of this program, one
can use \texttt{llvm-dis}, which produces a \texttt{.ll} file containing
the disassembled bitcode:

\begin{code}
llvm-dis aes128BlockEncrypt_driver.bc
\end{code}

Similarly, to view the disassembly after it has been transformed into
the LLVM-Sym representation, the \texttt{--xlate} option may be supplied
to \texttt{lss}.  This option causes the LLVM-Sym representation to be
displayed to stdout; the user may redirect output when convenient:

\begin{code}
lss --xlate aes128BlockEncrypt_driver.bc > aes128BlockEncrypt_driver.xlate
\end{code}

When viewing debugging output from \texttt{lss}, program locations are
currently shown in reference to the LLVM-Sym representation, so it is
sometimes useful to view that representation alongside \texttt{lss}
feedback.

\section{Verifying the Formal Model Using Cryptol}

One easy way to verify an LLVM implementation against a reference
specification is via the Cryptol tool set. Cryptol is a domain-specific
language created by Galois for the purpose of writing high-level but
precise specifications of cryptographic algorithms~\cite{cryptol}. The
Cryptol tool set has built-in support for checking the equivalence of
different Cryptol implementations, as well as comparing Cryptol
implementations to external formal models.

This tutorial comes with a handful of Cryptol files, most notably
\texttt{Rijndael.cry} and \texttt{equivAES.cry}.  The former is a
Cryptol specification of the Rijndael cipher.  In particular, it
contains the function \texttt{blockEncrypt} which should have equivalent
functionality to the \texttt{aes128BlockEncrypt} function in our C
source.  Well, nearly equivalent: we write a small wrapper around this
function, as can be seen in \texttt{equivAES.cry} that reorders the
bytes of the inputs and outputs as needed to the form expected by the
\texttt{blockEncrypt} function.  This essentially makes the calling
convention and data layout assumptions of both functions identical
before attempting to show equivalence.

To compare the functionality of the two implementations, we have several
options. As mentioned earlier, formal models can be evaluated on
concrete inputs, or compared to other formal models using proof
techniques to show equivalence for all possible inputs. The contents of
\texttt{equivAES.cry} show how to compare the formal model of the C
implementation against the Cryptol reference specification.

\begin{code}
...
extern AIG llvm_aes("../aes.aig") : ([4][32], [4][32]) -> [4][32];
theorem MatchesRef : {pt key}. llvm_aes (pt, key) == blockEncryptref_c (pt, key);
blockEncryptref_c : ([4][32], [4][32]) -> [4][32];
blockEncryptref_c (x, y) = ...
...
\end{code}

The \texttt{extern AIG} line makes the contents of \texttt{aes.aig}
available as a function called \texttt{llvm\_aes} that takes two
4x32-bit values as input and produces one 4x32-bit value as
output. Finally, the second line states a theorem: that the functions
\texttt{llvm\_aes} and \texttt{blockEncryptref\_c} should produce the
same ciphertext for all possible key and plaintext inputs.

We can load \texttt{equivAES.cry} into the Cryptol tool set, yielding
the following output:

\begin{code}


# cryptol equivAES.cry
Cryptol version 1.8.22, Copyright (C) 2004-2011 Galois, Inc.
                                            www.cryptol.net 
Type :? for help
Loading "equivAES.cry"..
  Including "Rijndael.cry"..
  Including "Cipher.cry"..
  Including "AES.cry".. Checking types..
  Loading extern aig from "../aes.aig".. Processing.. Done!
*** Auto quickchecking 1 theorem.
*** Checking "MatchesRef" ["equivAES.cry", line 5, col 1]
Checking case 100 of 100 (100.00%) 
100 tests passed OK
[Coverage: 0.00%. (100/11579208923731619542357098500868790785326998466564056...)]
\end{code}

By default, the Cryptol interpreter processes every \texttt{theorem}
declaration by automatically evaluating the associated expression on a
series of random values, and ensuring that it always yields ``true''.
In this case, it tried 100 random key and plaintext values, and the two
functions yielded the same output in each case. However, the number of
possible inputs is immense, so 100 test cases barely scratches the
surface.

To gain a higher degree of confidence that the functions do have the
same functionality for all possible inputs, we can attempt to prove
their equivalence deductively. From Cryptol's command line:

\begin{code}
equivAES> :set symbolic
equivAES> :prove MatchesRef
Q.E.D.
equivAES> :fm blockEncryptref_c "aes-ref.aig"
\end{code}

This tells the Cryptol interpreter to switch to symbolic simulation mode
(which is one way it can generate formal models from Cryptol functions)
and then attempt to prove the theorem named \texttt{MatchesRef}. On a
reasonably modern machine (as of September 2011), the proof should
complete in less than 30 minutes. The output \texttt{Q.E.D.} means that
the proof was successful.

Finally, the \texttt{:fm} command tells the interpreter to generate a
formal model of the \texttt{blockEncryptref\_c} function and store it in
the file \texttt{aes-ref.aig}. We can then use this formal model to
perform the same proof using an external tool such as ABC, as described
next.

Note that the above actions can be performed by running the
\texttt{check} target of the Makefile in the code subdirectory.

\section{Verifying the Formal Model Using ABC}

ABC is a tool for logic synthesis and verification developed by
researchers at UC Berkeley~\cite{abc}. It can perform a wide variety
of transformations and queries on logic circuits, including those in
the AIG form discussed earlier.

As an alternative approach to the equivalence check from the previous
section, we can use the \texttt{cec} command in ABC to attempt to
prove the model generated by the symbolic simulator equivalent to the
model generated from the Cryptol specification.

\begin{code}
# abc
UC Berkeley, ABC 1.01 (compiled Oct 26 2010 13:07:15)
abc 01> cec ./aes.aig ./aes-ref.aig
Networks are equivalent.
abc 01> 
\end{code}

% TODO: Add this section for final deliverable 
% \section{Evaluating Formal Models on Concrete Inputs}

% So far, we have demonstrated how to use the API of the symbolic
% simulator to generate formal models, in the form of And-Inverter
% Graphs, that describe the symbolic values of particular program
% variables, and then use external tools to analyze those formal models.
% The API also provides the ability to evaluate a formal model on
% specific concrete inputs from within the simulator.

% In the example from \texttt{JavaMD5.java}, the variable \texttt{out}
% depends on symbolic inputs and is therefore represented by a symbolic
% model. However, given concrete values for symbolic inputs that the
% output depends on, the model can be reduced to a concrete final value.
% Evaluation of a symbolic model to a concrete value uses one of the
% \texttt{Symbolic.evalAig} methods, depending on the type of the output
% variable of interest.

% Because the symbolic model describing an output variable may have
% inputs of various different types, the symbolic simulator API provides
% a class hierarchy to represent possible input values. The
% \texttt{CValue} class represents concrete input values, and has inner
% subclasses \texttt{CBool}, \texttt{CByte}, \texttt{CInt}, and
% \texttt{CLong}, to represent inputs of a variety of sizes.

% Given a model output variable and an array of concrete inputs, the
% \texttt{evalAig} function produces a new value of the same type as the
% given output variable, but one that is guaranteed to contain a
% concrete value from the perspective of the symbolic simulator. The
% caller of \texttt{evalAig} is responsible for providing the correct
% number of inputs. Otherwise, the simulator may throw a runtime error.

% For example, if we replace the call to \texttt{Symbolic.writeAiger}
% in the example above with the following code, the simulator will print
% out the result of evaluating the symbolic model on a concrete message.

% \begin{code}
% byte[] result = Symbolic.evalAig(out,
%   new CValue[] {
%     new CValue.CByte((byte) 0x68), // h
%     new CValue.CByte((byte) 0x65), // e
%     new CValue.CByte((byte) 0x6c), // l
%     new CValue.CByte((byte) 0x6c), // l
%     new CValue.CByte((byte) 0x6f), // o
%     new CValue.CByte((byte) 0x20), // 
%     new CValue.CByte((byte) 0x77), // w
%     new CValue.CByte((byte) 0x6f), // o
%     new CValue.CByte((byte) 0x72), // r
%     new CValue.CByte((byte) 0x6c), // l
%     new CValue.CByte((byte) 0x64), // d
%     new CValue.CByte((byte) 0x21), // !
%     new CValue.CByte((byte) 0x21), // !
%     new CValue.CByte((byte) 0x21), // !
%     new CValue.CByte((byte) 0x21), // !
%     new CValue.CByte((byte) 0x21)  // !
%   });
% for(int i = 0; i < result.length; i++) {
%   System.out.println(result[i]);
% }
% \end{code}

\bibliographystyle{plain}
\bibliography{bib/lss-tutorial}

\end{document}
