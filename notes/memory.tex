\documentclass{article}

\title{Symbolically Modeling the LLVM Memory Model}
\author{Galois, Inc\\Joe Hendrix\\\texttt{jhendrix@galois.com}}

% Who is the audience for this document?
% Sean and team members at Galois.
% What is the purpose?
% Explain some of the challenges in the LLVM memory, analyze different
% options for representing the memory, and recommend a specific model
% for Phase 1.

\begin{document}

\maketitle{}

\section{Summary}

LLVM bytecode is a low-level compiler intermediate representation that must
be sufficiently expressive to represent machine-specific low-level constructs
in C.  As such, it lacks the level of abstraction found in higher-level
typed byte code languages such as the JVM and~.Net.  This makes the design
of a suitable heap representation more tricky, as the memory model must
support the low-level features available to LLVM programs.

\section{Observations}

The text below includes some detailed observations made from reading
the LLVM Language Reference:

\begin{itemize}

  \item LLVM code has instructions for bitcasting from any type to any other
          type (either 'bitcast .. to' or a type specific instruction such as
          'inttoptr ... to').  It may be reasonable for conversions to fail
          (such as converting a symbolic integer to floating point), but even
          in that case we need to print a useful error message.

  \item Pointers are going to be 32 or 64bit integers (may want to make this an
          option). 16bit seems possible too, but may not be essential.

  \item Pointers may refer to basic block addresses (and consequently integers
          may as well).  However, the only code that will use this (in a
          defined way) is the 'indirectbr' instruction, which must also contain
          a list of possible targets.  Rather than have special values for
          these basic block addresses, it may be sufficient to simply maintain
          a mapping from (function,label) pairs to an associated address, then
          add conditional checks in handling basic blocks that checks for each
          potential target of 'indirectbr' whether the address associated to
          the target matches the value given to 'indirectbr'.


  \item A pointer may point to a function as well, and be used in a 'call' or
          'invoke'.  We probably want to require that a function pointer is
          ground, and maintain a map from valid integers to the function
          associated with that integer.

  \item The memory will need to refer to the following types of values:
  \begin{itemize}
    \item Global data
    \item Functions and individual basic blocks within a function.
    \item Variables allocated on the stack (using \texttt{alloca}).
    \item Variables allocated on the heap allocated using \texttt{malloc}, \texttt{new} in C++, 
            or an OS-specific memory allocation routine (e.g., \texttt{LocalAlloc}).
   \end{itemize}
\end{itemize}

Fundamentally, I think the heap should just be a map from integers to integers,
and we should rely on bitblasted bitpatterns to maintain distinctness (e.g.,
the heap could start at address 0x10000, while the stack is at address
0x20000).  The lookup function will take advantage of mismatches in the
high-order bits of allocations to differentiate between reads and writes.  We
could start with a slow linked list implementation, then look at more efficient
modes.

As a potential concern, we may run into problems if/when we switch to a word
level implementation, because the bit-patterns may not be as readily available.
Perhaps they could be generated on demand during the store or load instruction.

% LLVM First class types are the only type that can be produced by instructions.
% They include the following:
%   integer, floating point, pointer, vector, structure, array, label, metadata.
% * In addition to the variables below, all types have an 'undef' value and a
%   a 'trap value' (representing a undefined operation that may have side effects).

% Array
%   Values
%   * Constants
%   Operations
%   * 'extractvalue', 'inservalue', 
% Floating point
%   Values
%   * Constants with one the following subtypes (see [1] for more details)
%     float	 32-bit floating point value
%     double	 64-bit floating point value
%     fp128	 128-bit floating point value (112-bit mantissa)
%     x86_fp80	 80-bit floating point value (X87)
%     ppc_fp128 128-bit floating point value (two 64-bits)
%   * Constants formed from following operations:
%     'fadd', 'fdiv', 'fmul', 'frem', 'fsub'
%   * 'fptrunc', 'fpext'
%   * 'uitofp', 'sitofp' | Convert integer to floating point, may fail if
%       integer is symbolic.
% Function
%   Values
%   * zero?
%   Operations
%   * 'call', 'invoke'
% Integer | iX (where X is the bit width).
%   A simple type that specifies an arbitrary bit width for the integer type.
%   Values:
%   * 'true' and 'false' for i1.
%   * Integer constants
%   * Integer variables
%   * Binary operations applied to two integers:
%     'add', 'mul', 'sdiv', 'srem 'sub', 'udiv', 'urem'
%   * Bitwise inary operations applied to two integers:
%     'shl', 'lshr', 'ashr', 'and', 'or', 'xor', 
%   * 'sext', 'trunc', 'zext'
%   * 'fptoui', 'fptosi' | Convert floating point to integer.
% Label
%   A pointer to a basic block.  It is referenced in branch instructions.
%   Values
%   * Labels with a block.
%   Operations
%   * Branch to specific blocks.
% Metadata
%   Values
%   * Structure like constants
%   Operations
%   * ??? Note sure
% Opague
%   Structure types that do not have a body specified (such as a C forward
%   declared struct).
%   * Just used for pointer conversions (afaik).
% Pointer | <type> *
%   Specifies code and data in memory.  May have an operational address space
%     attribute.
%   Values
%   * null
%   * Basic block Labels "i8* blockaddress(@fn, %label)"
%   * Global variables and function addresses.
%   * Stack and heap allocations.
%   Operations
%   * 'alloca', 'load', 'store'
%   * 'cmpxchg', 'atomicrmw', 'getelementptr'
%   * 'ptotoint' | Convert a pointer to an integer type.
%   * 'inttoptr' | Convert integer to pointer type.
% Structure
%   Values
%   * Constants
%   Operations
%   * 'extractvalue', 'inservalue', 
% Vector | <# elements> x <elementtype>
%   A non-empty vector of elements (used with SIMD instructions).
%   Values
%   * Constants
%   * Pointwise arithmetic and bitwise binary operations on the underlying
%     type (e.g., 'add').
%   Operations
%   * 'extractelement', 'insertelement', 'shufflevector'
% X86mmx
%   Values
%   * Zero (possibly?)
%   Operations?? Unclear to me.
% 
% [1] http://llvm.org/docs/LangRef.html

\section{Representation Options}

TODO: Describe some high-level issues.


\subsection{Concrete Address Representation}

In the concrete address representation, the heap is modeled as a data structure
mapping concrete addresses to symbolic values.  The data structure itself could
be a common data structure such as a binary tree or finger tree.  Reads and writes
of concrete addresses have the same relative complexity measures as a purely
concrete heap would, albeit with a some overhead to deal with the symbolic values
stored in the heap.

Reads and writes to symbolic addresses with this representation have
considerably more overhead than the concrete case.  A read will need to
construct a term that selects among all the possible  appropriate value in the 

reading and writing to concrete addresses. 




TODO: Describe an eager bit-level representation where writes are
pushed to the individual bits as soon as they occur.

\end{document}
